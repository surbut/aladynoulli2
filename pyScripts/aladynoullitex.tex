% LaTeX template for Science journal paper
% Based on science_template.tex with adaptations for research content

%%%%%%%%%%%%%%%% START OF PREAMBLE %%%%%%%%%%%%%%%

% Basic setup. Authors shouldn't need to adjust these commands.
% They need to be included in this .tex for Science's production software to work.

% Use the basic LaTeX article class, 12pt text
\documentclass[12pt]{article}

% Science uses Times font. If you don't have this installed, comment out the following line
\usepackage{newtxtext,newtxmath}
% Depending on your LaTeX fonts installation, you might get better results with one of these:
%\usepackage{mathptmx}
%\usepackage{txfonts}

% Allow external graphics files
\usepackage{graphicx}

% Use US letter sized paper with 1 inch margins
\usepackage[letterpaper,margin=1in]{geometry}

% Double line spacing, including in captions
\linespread{1.5} % For some reason double spacing is 1.5, not 2.0!

% One space after each sentence
\frenchspacing

% Abstract formatting and spacing - no heading
\renewenvironment{abstract}
	{\quotation}
	{\endquotation}

% No date in the title section
\date{}

% Reference section heading
\renewcommand\refname{References and Notes}

% Figure and Table labels in bold
\makeatletter
\renewcommand{\fnum@figure}{\textbf{Figure \thefigure}}
\renewcommand{\fnum@table}{\textbf{Table \thetable}}
\makeatother

% Call the accompanying scicite.sty package.
% This formats citation numbers in Science style.
\usepackage{scicite}

% Provides the \url command, and fixes a crash if URLs or DOIs contain underscores
\usepackage{url}

%%%%%%%%%%%% CUSTOM COMMANDS AND PACKAGES %%%%%%%%%%%%

% Authors can define simple custom commands e.g. as shortcuts to save on typing
% Use \newcommand (not \def) to avoid overwriting existing commands.
\newcommand{\aladyn}{\texttt{ALADYNOULLI}}

%%%%%%%%%%%%%%%% TITLE AND AUTHORS %%%%%%%%%%%%%%%%

% Title of the paper.
% Keep it short and understandable by any reader of Science.
% Avoid acronyms or jargon. Use sentence case.
\def\scititle{
	\aladyn: A Bayesian approach to disease progression modeling for genomic discovery and clinical prediction
}
% Store the title in a variable for reuse in the supplement (otherwise \maketitle deletes it)
\title{\bfseries \boldmath \scititle}

% Author and institution list.
% Institution numbers etc. should be hard-coded, do *not* use the \footnote command.
\author{
    Sarah M. Urbut$^{1,2\ast}$,
    Yi Ding$^{3}$,
    Xilin Jiang$^{4}$,
    Tet Nakao,
    Anika Misra,\and
    Whitney Hornsby$^{1,2}$,
    Jordan Smoller$^{1,2}$,\and
    Alexander Gusev$^{5,6}$,
    Giovanni Parmigiani$^{3,5\dagger}$,
    Pradeep Natarajan$^{1,2,7\dagger}$\and
    % Institution list, in a slightly smaller font
    \small$^{1}$Division of Cardiology, Department of Medicine, Massachusetts General Hospital, Boston, MA 02114, USA\and
    \small$^{2}$Cardiovascular Research Center, Massachusetts General Hospital, Boston, MA 02114, USA\and
    \small$^{3}$Department of Data Sciences, Dana-Farber Cancer Institute, Boston, MA 02215, USA\and
    \small$^{4}$Big Data Institute, Li Ka Shing Centre for Health Information and Discovery, University of Oxford, Oxford OX3 7LF, UK\and
    \small$^{5}$Department of Biostatistics, Harvard TH Chan School of Public Health, Boston, MA 02115, USA\and
    \small$^{6}$Department of Medical Oncology, Dana-Farber Cancer Institute, Boston, MA 02215, USA\and
    \small$^{7}$Broad Institute of Harvard and MIT, Cambridge, MA 02142, USA\and
    % Identify at least one corresponding author, with contact email address
    \small$^\ast$Corresponding author. Email: surbut@mgh.harvard.edu\and
    % Joint contributions can be indicated like this
    \small$^\dagger$These authors jointly supervised this work.
}
%%%%%%%%%%%%%%%%% END OF PREAMBLE %%%%%%%%%%%%%%%%


%%%%%%%%%%%%%%%% START OF MAIN TEXT %%%%%%%%%%%%%%%
\begin{document} 

% Insert the title and author list
\maketitle

% Abstract, in bold
\begin{abstract} \bfseries \boldmath
Understanding how disease risk evolves over a lifetime is a central challenge in medicine. While electronic health records (EHRs) provide rich longitudinal data, most existing models analyze each disease in isolation, missing the complex interplay between multiple conditions and genetic factors. Here, we introduce $\aladyn$, a Bayesian framework that uncovers latent, clinically meaningful disease signatures from longitudinal health records, while simultaneously modeling personalized, time-resolved risk trajectories. Applied to over 400,000 individuals across three major biobanks with up to 50 years of follow-up, $\aladyn$ robustly identifies 20 interpretable disease signatures that are remarkably consistent across diverse populations and healthcare systems. These signatures reveal distinct, biologically grounded patterns of disease progression, demonstrate strong and specific genetic correlations, and enable both accurate prediction of patient risk and discovery of novel genetic associations. By seamlessly integrating genetic data with dynamic, multi-disease risk modeling, $\aladyn$ provides a unified, interpretable, and clinically actionable approach to understanding disease comorbidity, empowering precision medicine and opening new avenues for therapeutic discovery.
\end{abstract}



% The first paragraph does NOT have a heading
\noindent
\noindent
Disease risk is not static; it evolves dynamically over the human lifespan, shaped by a complex interplay of genetic predisposition, environmental exposures, and the accumulation of comorbidities. Capturing these individualized, time-varying patterns of risk is essential for transforming early detection, prevention, and personalized treatment strategies \cite{berry2006bayesian, angus2021heterogeneity}. The recent proliferation of large-scale EHRs linked to genetic data provides an unprecedented opportunity to model disease trajectories at population scale \cite{sudlow2015biobank, pedersen2023adult}. Yet, extracting actionable insights from these rich, longitudinal datasets remains a formidable challenge, due to the heterogeneity of patient populations, the temporal complexity of disease progression, and the intricate relationships among diverse conditions.

Traditional analytic approaches are fundamentally limited: they focus on isolated diseases or simple pairwise associations, failing to capture how multiple conditions co-evolve over time \cite{wang2021empirical}. Even recent unsupervised methods, while able to identify disease clusters or trajectories \cite{jiang_age-dependent_2023}, typically ignore the temporal dynamics of risk and the profound influence of genetic factors on disease progression \cite{urbut_dynamic_2023, hyttinen_genetic_2003}. Most models assume conditional independence of diseases, missing the opportunity to leverage shared information across related conditions for both prediction and discovery \cite{blei2003latent, blei2006dynamic}.

To overcome these limitations, we developed $\aladyn$, a probabilistic model that integrates genetic data with longitudinal EHRs to reveal latent disease signatures and model individual-specific health trajectories over time. Our approach combines Gaussian processes with a flexible mixture model, capturing the temporal evolution of disease risk, learning shared patterns of disease co-occurrence, and allowing for rich individual variation. By jointly modeling multiple diseases and their genetic determinants, $\aladyn$ enables both state-of-the-art prediction of future disease risk and powerful discovery of the genetic architecture underlying complex phenotypes.    
% Research Articles split the text into sections using descriptive headings

\subsection*{Model framework captures temporal disease signatures and individual trajectories}

Disease patterns among individuals vary by onset, progression speed, and composition, reflecting different underlying biological mechanisms. In contrast to topic models that conditionally allocate observed diseases to categories \cite{jiang_age-dependent_2023, blei2006dynamic}, $\aladyn$ models the probability of each disease for an individual by integrating across multiple latent signatures (Fig. 1A).

For each individual $i$, disease $d$, and time point $t$, we model the probability of disease occurrence $\pi_{idt}$ as:

\begin{equation}
\pi_{idt} = \kappa \cdot \sum_{k=1}^{K} \theta_{ikt} \cdot \text{sigmoid}(\phi_{kdt})
\end{equation}

\noindent where $\kappa$ is a global calibration parameter, $\theta_{ikt}$ represents individual $i$'s time-varying association with signature $k$ at time $t$, and $\phi_{kdt}$ captures the relationship between signature $k$ and disease $d$ over time.

The individual-signature associations $\theta_{ikt}$ are derived from latent variables $\lambda_{ikt}$ through a softmax transformation:

\begin{equation}
\theta_{ikt} = \frac{\exp(\lambda_{ikt})}{\sum_{k'=1}^{K} \exp(\lambda_{ik't})}
\end{equation}

These latent variables follow a Gaussian process prior with a mean function informed by genetic factors:

\begin{equation}
\lambda_{ik} \sim \mathcal{GP}(r_k + \mathbf{g}_i^{\top}\Gamma_k, K_{\lambda})
\end{equation}

\noindent where $r_k$ is a signature-specific reference level, $\Gamma_k$ captures genetic effects on signature affinity, $\mathbf{g}_i$ represents individual genetic factors, and $K_{\lambda}$ is a temporal covariance kernel ensuring smooth trajectories.

Similarly, the disease-signature associations follow a Gaussian process prior:

\begin{equation}
\phi_{kd} \sim \mathcal{GP}(\mu_d + \psi_{kd}, K_{\phi})
\end{equation}

\noindent where $\mu_d$ is a disease-specific baseline reflecting population prevalence, $\psi_{kd}$ represents the overall strength of association between signature $k$ and disease $d$, and $K_{\phi}$ allows for temporal variation in these associations.

Our model has several key innovations compared to existing approaches. First, we employ a Bernoulli discrete-time survival likelihood rather than an allocation framework, enabling direct prediction of disease risk instead of modeling conditional probabilities given disease occurrence. Second, our mixture formulation allows multiple signatures to simultaneously contribute to disease risk, reflecting the reality that diseases often arise from multiple underlying biological processes. Third, by modeling both individual trajectories and signature-disease associations as Gaussian processes, we capture smooth temporal variations in risk. Finally, our incorporation of genetic factors allows these signatures to have biological interpretations informed by genetic architecture.


A key innovation of our approach lies in its formulation as a mixture of probabilities rather than a probability of a mixture. Unlike allocation-based topic models that conditionally assign diseases to topics after they've occurred [citation to ATM paper], ALADYNOULLI directly models the probability of disease occurrence as a weighted combination of signature-specific probabilities.

This crucial distinction allows our model to: (1) predict future disease onset rather than merely explain observed diagnoses; (2) accommodate multiple contributing disease processes simultaneously rather than forcing competitive allocation to a single signature; and (3) accurately model chronic conditions that persist over time rather than treating each diagnosis as an independent event. The combination of softmax-transformed individual loadings ($\theta$) and sigmoid-transformed disease probabilities ($\phi$) ensures proper probability scaling while maintaining model identifiability without arbitrary constraints.

To evaluate predictive performance, we employed a rigorous temporal validation framework that prevents data leakage. While the full model was trained using complete patient histories, predictions were generated using only information available up to an enrollment time point. This approach simulates real-world clinical scenarios where physicians must predict future risk based solely on a patient's history to date, ensuring our performance metrics reflect true predictive capability rather than retrospective explanation.

% Next heading

\subsection*{Discovery of robust, clinically meaningful disease signatures}

We applied $\aladyn$ to three independent cohorts: UK Biobank (UKB, n=407,878), Mass General Brigham (MGB, n=47,327), and All of Us (AoU, n=94,158). Despite differences in population characteristics, healthcare systems, and data collection methodologies, the model identified 20 distinct, clinically interpretable disease signatures that were remarkably consistent across cohorts (Fig.~2, Table~1). Each signature captured a unique constellation of related conditions—such as cardiovascular, metabolic, pulmonary, psychiatric, and malignancy signatures—each with characteristic temporal patterns of risk.

These signature patterns were not only robust across populations, but also biologically meaningful. For example, the cardiovascular signature showed steadily increasing risk for atrial fibrillation and heart failure after age 55, while the malignancy signature displayed a sharp rise in metastatic disease probabilities between ages 60 and 75. The model also captured nuanced temporal relationships, such as the earlier peak of type 1 diabetes compared to type 2 diabetes, and the progression from primary to metastatic cancer.

The consistency of these signatures across biobanks is illustrated in Fig.~2E, which shows that the temporal patterns of key diseases within each signature remain stable despite differences in healthcare systems and coding practices. This robustness supports the biological validity of the discovered patterns.

\subsection*{Personalized trajectories reveal heterogeneity within disease categories}

Beyond population-level patterns, $\aladyn$ provides individualized, time-resolved risk trajectories. By modeling each person’s evolving association with different disease signatures, the model reveals profound heterogeneity within clinical diagnoses. For example, among patients with major depressive disorder, $\aladyn$ identified subgroups dominated by psychiatric/fibromyalgia, musculoskeletal/gastrointestinal, or cardiovascular signatures (Fig.~3D). Similarly, breast cancer patients exhibited distinct signature profiles reflecting inflammatory, hormonal, or metabolic contributions.

This ability to resolve patient subgroups with distinct biological underpinnings has immediate implications for precision medicine. For coronary atherosclerosis, we found that early-onset cases (under 50 years) exhibited sharper increases in cardiovascular signature loadings and stronger genetic associations than late-onset cases (over 70 years), suggesting different etiological pathways and informing targeted prevention strategies (Fig.~3C).

\subsection*{Genetic architecture and signature-based discovery}

A defining strength of $\aladyn$ is its seamless integration of genetic data. By modeling genetic effects on signature activations, the model enables both improved risk prediction and novel biological discovery. We quantified genetic associations with signature trajectories by computing the area under each individual’s signature curve (“lifetime signature exposure”) and performing genome-wide association studies (GWAS) (Fig.~4A). This approach identified both known and novel genetic loci, including variants that associate more strongly with signature loadings than with any individual component disease.

Polygenic risk scores (PRS) for cardiovascular, metabolic, and psychiatric diseases showed strong, specific associations with their corresponding signatures (Fig.~4B). Genetic correlation analysis revealed shared architecture between signatures and related complex traits (Fig.~4C). Notably, signature-based GWAS uncovered loci uniquely associated with shared disease processes, highlighting the power of this approach to reveal biology that is invisible to single-disease analyses (Fig.~4D).


\subsection*{Dynamic risk assessment improves disease prediction}

A primary motivation for modeling longitudinal disease patterns is to improve prediction of future disease events. We evaluated $\aladyn$'s predictive performance using a temporal validation framework, where we predicted disease onset within one year following a defined enrollment time point, using only data available up to enrollment.

We compared $\aladyn$ to a strong baseline Cox proportional hazards model that included age, sex, and family history covariates. Figure \ref{fig:performance}A shows the area under the receiver operating characteristic curve (AUC) for 1-year and 10-year prediction across 28 major disease categories. For 1-year prediction, $\aladyn$ demonstrated significantly improved performance compared to the baseline model for most conditions, with particularly large improvements for conditions including rheumatoid arthritis (+0.39), colorectal cancer (+0.33), breast cancer (+0.29), Parkinson's disease (+0.27), and anxiety disorders (+0.24).

The model also showed excellent calibration for 1-year risk predictions (Fig. \ref{fig:performance}C), with predicted probabilities closely matching observed event rates across the risk spectrum. This calibration is crucial for clinical applications, ensuring that predicted risk levels accurately reflect patients' true likelihood of experiencing disease events.

To demonstrate $\aladyn$'s ability to capture changing risk dynamics, we examined how signature activations evolve preceding disease onset. Figure \ref{fig:performance}B shows examples of signature loading trajectories for patients diagnosed with myocardial infarction, revealing increases in cardiovascular signature activation beginning approximately 2-3 years before the clinical event. Importantly, these patterns emerge even when the target disease itself is censored from the input data, indicating the model captures informative signals from related comorbidities.

These results highlight $\aladyn$'s strength in providing dynamically updated risk assessments that incorporate both individual history and population-level patterns. The model's ability to identify subtle shifts in signature activations before clinical manifestations offers potential for earlier intervention in high-risk individuals.


\section*{Discussion}

\subsection*{Implications for precision medicine and therapeutic targeting}

By revealing distinct, biologically grounded patient subgroups and providing dynamic, individualized risk profiles, $\aladyn$ opens new frontiers in precision medicine. Signature profiles can inform targeted prevention and treatment strategies, identify patients most likely to benefit from specific interventions, and enable dynamic adjustment of risk as new clinical information accrues. Signature-based patient stratification can also enhance clinical trial efficiency by identifying more homogeneous populations and responder subgroups.


\aladyn represents a paradigm shift in the modeling of disease risk and progression. By unifying genetics and longitudinal phenotyping in a single, interpretable framework, our approach provides a powerful new lens for understanding disease comorbidity, predicting future events, and discovering the genetic architecture of complex traits. The model’s identification of robust, clinically meaningful signatures across diverse populations supports their biological validity and translational relevance. Its ability to resolve patient heterogeneity, deliver dynamic, calibrated risk predictions, and uncover novel genetic associations positions $\aladyn$ as a foundational tool for the next generation of precision medicine.

While our model relies on EHR data, which may contain biases related to healthcare access and coding practices, and does not yet explicitly model environmental or lifestyle factors, these limitations are shared by all large-scale health data analyses. Future work integrating additional data sources will further enhance predictive performance and biological insight.



%%%%%%%%%%%%%%%% MAIN TEXT FIGURES %%%%%%%%%%%%%%%

\begin{figure} % Do NOT use \begin{figure*}
	\centering
	\includegraphics[width=1\textwidth]{figures/fig1.pdf} % Model diagram
	% Pick an appropriate width

	% Captions go below figures
	\caption{\textbf{ALADYNOULLI model overview and probabilistic framework.}
		(\textbf{A}) Schematic representation of the model showing how latent disease signatures and individual trajectories interact over time. Disease probabilities ($\pi_{idt}$) are generated as a mixture of signature-specific contributions, with each signature's importance to an individual ($\lambda_{ikt}$) evolving over time based on genetic factors and accumulated diagnoses. (\textbf{B}) Probabilistic graphical model showing relationships between observed diagnoses ($Y_{idt}$), disease probabilities ($\pi_{idt}$), individual signature loadings ($\theta_{ikt}$), and latent variables ($\lambda_{ik}$, $\phi_{kd}$). (\textbf{C}) Comparison with traditional allocation models, illustrating how $\aladyn$ allows multiple signatures to simultaneously contribute to disease risk rather than forcing competitive allocation. (\textbf{D}) Example of temporal evolution of individual risk incorporating new diagnostic information over the life course.}
	\label{fig:model_overview}
\end{figure}

\begin{figure} % Do NOT use \begin{figure*}
	\centering
	\includegraphics[width=0.9\textwidth]{figures/Fig2_May7.pdf} % Population-level signatures
	% Pick an appropriate width

\caption{
\textbf{Population-level disease signatures and temporal patterns inferred by \aladyn.}
(\textbf{A}) Age-dependent log hazard ratios for four representative disease signatures (cardiovascular, cancer, pulmonary, and colorectal), as estimated by the model. Each line represents the predicted risk trajectory for a specific disease within the signature, illustrating distinct temporal patterns of disease onset.
(\textbf{B}) Heatmap of signature-disease specificity parameters ($\psi_{kd}$) learned by the model, with red indicating strong positive association and blue indicating negative association between diseases and signatures.
(\textbf{C}) Cluster correspondence matrices comparing model-inferred disease groupings across biobanks (UK Biobank, MGB, and All of Us), demonstrating the consistency of disease clusters for common diseases.
(\textbf{D}) Model-predicted age-specific probabilities of disease onset for a range of conditions, showing the temporal emergence of diseases across the lifespan.
(\textbf{E}) Comparison of signature trajectories for cardiovascular and malignancy signatures across three independent biobanks (MGB, AoU, UKB), demonstrating the robustness and reproducibility of the model’s temporal patterns across cohorts.
}
\label{fig:signature-dynamics}
\end{figure}

\begin{figure}
	\centering
	\includegraphics[width=1\textwidth]{figures/Fig3_May9.pdf}
	\caption{\textbf{Individual-level trajectories and dynamic risk profiles.}
		(\textbf{A}) Patient-specific signature loadings ($\theta$) over time for a representative individual, illustrating how the contributions of different latent signatures evolve as new diagnoses are acquired (diagnosis times indicated by stars on the timeline). The right panel summarizes the patient's average signature composition across all time points. (\textbf{B}) Decomposition of myocardial infarction (MI) risk: Top, time-varying signature loadings and their contributions to MI risk; middle, heatmap of log disease probabilities by signature and age; bottom, stacked area plot showing the aggregate risk over time. Right, signature-specific deviations from the population reference for MI. (\textbf{C}) Comparison of early-onset ($<$55 years) and late-onset ($>$70 years) MI: Average signature loadings and their temporal velocities reveal distinct dynamic patterns and rates of change associated with age of onset. (\textbf{D}) Signature heterogeneity within disease subtypes: Stacked area plots show deviations in signature proportions from the population average for selected diseases, highlighting the diversity of underlying biological processes among patients with the same clinical diagnosis.
        }
	\label{fig:individual-trajectories}
\end{figure}

\begin{figure}
	\centering
	\includegraphics[width=1\textwidth]{figures/fig4.pdf} % Updated figure name
\caption{
\textbf{Genetic architecture of disease signatures and their component phenotypes.}
(\textbf{A}) Schematic of the genetic association approach: time-varying signature loadings ($\lambda$) inferred by the ALADYNOUILLI model are summarized as area under the curve (AUC, or "lifetime signature exposure") for each individual, which is then used as a quantitative phenotype for genome-wide association studies (GWAS).
(\textbf{B}) Heatmaps showing segregation of polygenic risk scores (PRS) for selected diseases across model-inferred disease clusters, illustrating how genetic risk is distributed among signature-defined subgroups.
(\textbf{C}) Heatmap of positive genetic correlations (LDSC $r_g$) between signature exposures and a range of complex diseases and traits, with asterisks indicating statistical significance ($^{*}p < 0.05$, $^{**}p < 0.01$).
(\textbf{D}) UpSet plot comparing the overlap of genome-wide significant loci identified in signature-based GWAS versus GWAS of individual component diseases, highlighting loci uniquely discovered by the signature approach.
The Manhattan plot in panel (A) illustrates the GWAS performed on the signature exposure phenotype, demonstrating the discovery potential of this approach.
}
\label{fig:genetics}
	\label{fig:genetics}
\end{figure}

\begin{figure}
	\centering
	\includegraphics[width=1.5\textwidth]{figures/fig5.pdf} % Updated figure name
\caption{
\textbf{Predictive performance and risk dynamics of the ALADYNOULLI model.}
(\textbf{A}) Comparison of 10-year prediction performance (AUC) for the ALADYNOUILLI model versus established risk models across multiple diseases. For ASCVD, the pooled cohort equation and PREVENT models are shown; for breast cancer, the Gail model; for other diseases, a Cox survival model with age, sex, and family history (when available) is used as baseline. Event rates and 95\% confidence intervals are indicated for each disease.
(\textbf{B}) Signature (softmax) trajectory patterns: upper panel shows mean signature 5 (cardiovascular) trajectories for myocardial infarction (MI) cases versus controls; lower panel displays individual patient trajectories, with diagnosis times (blue) and censoring times (red) marked.
(\textbf{C}) Calibration plot for model predictions at enrollment, comparing predicted and observed event rates across the full range of risk, demonstrating strong model calibration.
(\textbf{D}) ROC curves for 10-year ASCVD prediction, comparing the ALADYNOUILLI model to baseline and Cox models, with AUC values indicated.
(\textbf{E}) Time-dependent AUC for ASCVD prediction, showing model discrimination performance as a function of follow-up time.
}
\label{fig:performance}
\end{figure}



%%%%%%%%%%%%%%%% REFERENCES %%%%%%%%%%%%%%%

\clearpage % Clear all remaining figures and tables then start a new page

% The list of references goes after the main text and before the acknowledgements
% When preparing an initial submission, we recommend you use BibTeX, like this:
%
\bibliography{science_template} % for a file named science_template.bib
\bibliographystyle{sciencemag}

% After the paper has completed peer review and been revised ready for acceptance,
% you should comment out the lines above and copy-paste the contents of your .bbl
% file here instead. This will help ensure that our conversion software works correctly.

%%%%%%%%%%%%%%%% ACKNOWLEDGEMENTS %%%%%%%%%%%%%%%

\section*{Acknowledgments}
\paragraph*{Funding:}
This work was supported by National Institutes of Health grants (R01HL155915, R01HL157635, R35HL144758) to P.N., American Heart Association grants (19SFRN34800000, 19SFRN34850009) to P.N.
\paragraph*{Author contributions:}
S.M.U., PN and GP conceptualized the study, developed the methodology, implemented the software, and wrote the original draft. Y.D. and XJ contributed to methodology development and formal analysis. W.H. assisted with data curation and visualization. A.G., P.N., and G.P. provided supervision, resources, and critical review. All authors contributed to manuscript review and editing.
\paragraph*{Competing interests:}
The authors declare no competing interests.
\paragraph*{Data and materials availability:}
The code for implementing $\aladyn$ is available at https://github.com/surbut/aladynoulli2. Access to individual-level UK Biobank data requires approval from the UK Biobank (https://www.ukbiobank.ac.uk/). Access to Mass General Brigham data requires approval from the Mass General Brigham Institutional Review Board. Access to All of Us data requires approval through the All of Us Researcher Workbench (https://www.researchallofus.org/).

%%%%%%%%%%%%%%%% SUPPLEMENT LIST %%%%%%%%%%%%%%%

\subsection*{Supplementary materials}
Materials and Methods\\
Supplementary Text\\
Figs. S1 to S10\\
Tables S1 to S4\\
References \textit{(30-42)}\\
Data S1

%%%%%%%%%%%%%%%% END OF MAIN TEXT %%%%%%%%%%%%%%%

\end{document}